\section{Cleaning data for analysis}

% What is data cleaning
\textbf{Data cleaning} is a widely used expression used to refer to different tasks.
The cleaning process, as defined in this book, involves
(1) making the dataset easy to use and understand, and 
(2) carefully exploring each variable to document their distributions and identify patterns that may bias the analysis.
The resulting dataset will contain only the variables collected in the field, and
no modifications to data points will be made, 
except for corrections of mistaken entries.
Apart from the \textbf{cleaned dataset} (or datasets) itself,
cleaning will also yield extensive documentation describing  it.

% Section overview
During data cleaning, you will acquire in-depth understanding of the contents and structure of your data.
This knowledge will be key to correctly construct final indicators and analyze them.
So don't rush through this step.
Explore the dataset using tabulations, summaries, and descriptive plots.
It is common for cleaning to be the most time-consuming task in a project.
In this section, we will introduce some concepts and tools to make it more efficient and productive.

%\subsection{Correcting data points}

\subsection{Recoding and annotating data}

